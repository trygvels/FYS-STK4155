\documentclass[10pt, a4paper]{amsart}

\usepackage[]{graphicx}
\usepackage{subcaption}
\usepackage[toc,page]{appendix}
\usepackage[]{hyperref}
\usepackage[]{physics}
\usepackage[]{listings}
\usepackage[utf8]{inputenc}
\usepackage[dvipsnames]{xcolor}
\usepackage{dirtytalk}

\definecolor{mygray}{gray}{0.9}

\lstset{
	frame = single,
	language = Python,
	showstringspaces = false,
	tabsize = 2,
	otherkeywords = {self},
	keywordstyle = \color{Maroon},
	identifierstyle=\color{olive},
 	stringstyle=\color{orange},
 	backgroundcolor=\color{mygray},
 	breaklines = true
}

\title[Regression and Resampling]{Regression and Resampling} \\
\normalsize{Project 1}\hrulefill\small{ FYS-STK4155: Applied data analysis and machine learning}\hrulefill}

\author[Svalheim]{Trygve Leithe Svalheim \\
  \href{https://github.com/trygvels/}{\texttt{github.com/trygvels}}}
  
\date{\today}

\begin{document}

\begin{titlepage}
\begin{abstract}

\end{abstract}
\maketitle
\tableofcontents
\end{titlepage}

%% -----------------
%% INTRODUCTION
\section{Introduction}
Talk about the what Regression is, and the importance of regularization and resampling.

%% -----------------
\section{Bias-Variance, Underfitting and overfitting, regularization}

\section{Regression methods}

\subsection{OLS}
\subsection{Ridge}
\subsection{Lasso}

%% -----------------
%% RESULTS
\section{Comparison}
Fitting to the Franke function



\begin{figure}
	\centering
	\includegraphics[width=0.9\textwidth]{../figs/biasvariance.png}
	\caption{Stuff}
	\label{fig:distribution}
\end{figure}



%\begin{figure}
%\begin{subfigure}{.5\textwidth}
%  \centering
%  \includegraphics[width=\textwidth]{../figures/5ac/5cLOGLOG_N500_varSavings.pdf}
%  \caption{}
%  \label{fig:logdistribution}
%\end{subfigure}%
%\begin{subfigure}{.5\textwidth}
%  \centering
%  \includegraphics[width=\linewidth]{../figures/5ac/omega.pdf}
%  \caption{}
%  \label{fig:straightline}
%\end{subfigure}
%\caption{Logarithmic plot of wealth distribution from figure \ref{fig:distribution} is shown in figure \ref{fig:logdistribution}. Gibbs measure (equation \ref{eq:gibbs}) of every data point from the simulations plotted with logarithmic $y$-axis in figure \ref{fig:straightline}.}
%\label{fig:logplot}
%\end{figure}




%% ----------------
%% DISCUSSION
\section{Discussion}
		
%% ---------------
%% CONCLUSION
\section{Summary Remarks}


\begin{thebibliography}{10}

\bibitem{Patriarca} Patriarca, M., Chakraborti, A., \& Kaski, K. (2004). 
	Gibbs versus non-Gibbs distributions in money dynamics. 
	\emph{Physica A: Statistical Mechanics and its Applications},
	340(1), pp. 334-339.
\end{thebibliography}

\vfill
\pagebreak

\begin{appendices}

\section{Parametrization from Patriarca et al.}
\label{app:partriarca}
A corresponding exact solution of an income distributions for a generic value of $\lambda$ with $0<\lambda<1$ is provided in Patriarca et al.\cite{Patriarca}. Fir one employs the reduced variable
\begin{equation}
x = \frac{m}{\ev{m}}
\end{equation}
the agent money in units of average money $\ev{m}$ and the parameter
\begin{equation}
n(\lambda) = 1 + \frac{3\lambda}{1-\lambda}.
\end{equation}
The money distributions, for arbitrary values of $\lambda$, are well fitted by the function
\begin{equation}
P_n(x) = a_nx^{n-1}e^{-nx}
\end{equation}
where $a_n$ is a normalization factor shown to be
\begin{equation}
a_n = \frac{n^n}{\Gamma(n)}
\end{equation}
\end{appendices}

\end{document}
